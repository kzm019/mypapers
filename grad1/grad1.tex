\documentclass[12pt, a4paper]{jsarticle}
\usepackage[dvipdfmx]{graphicx}
\usepackage{url}
\usepackage{listings}
\usepackage{amsmath,amssymb}
\usepackage{mathtools,amssymb}
\usepackage{bm}
\usepackage{ascmac}
\usepackage{graphicx}
\usepackage[dvipdfmx]{color}
\usepackage{tikz}
\usepackage[version=4]{mhchem}
\usepackage{xcolor}
\usepackage{here}
\usepackage{wrapfig}
\usepackage{mathrsfs}
\usetikzlibrary{intersections,calc,arrows.meta}

\title{ゼミ発表用ノートver.1}
\author{中越一磨}
\begin{document}
\maketitle

\section*{はじめに}
このノートは,ゼミ発表の補助として用いるために書いている.煩雑な計算をゼミ内ですべて触れることは難しいので適宜このノートを参照していただきたい.なお,各回でノートを作っていく形式ではなく更新していく形を取るので,下記途中や中途半端に書き残しているところなどお見苦しいところはあるがお許しください.

ここで,発表内容や方針などについて触れておきたい.私の卒業研究のテーマは第1回でも触れた通り,「4次元空間内での重力場方程式のEinstein方程式の唯一性」や「Gauss-Bonnet termのような高次曲率項についての考察」などとなる.そこで,これらのことを学ぶ上で以下の論文を読んでいこうと予定している.\footnote{main論文などについては今後の進捗状況に応じて変更・追加する可能性がある.}
\begin{enumerate}
  \item (main) The Einstein Tensor and its Generalizations
  \item The Uniqueness of the Einstein Field Equations in a Four-Dimensional Space
\end{enumerate}

順序としては,第2回では2.の論文を扱い,第3回で1.の論文について扱い,今後の状況に応じて第4回で1.の論文か4次元に話を絞りGauss-Bonnet項についての話題のみとするかを予定している.
\part{The Uniqueness of the Einstein Field Equations in a Four-Dimensional}
\section{Introduction}
この論文は,Einstein方程式が4次元では唯一の方程式になる事実を示したものである.
\subsection{場の解析力学}
前期にも扱いましたがLagrangianの変数を場の変数にしたときのEuler-Lagrange方程式\footnote{以後EL方程式と省略}について少し解説する.今ここでは一般に場の変数を\(\Psi^A = \Psi^A(x^i)\)として,Lagrangianは
\begin{equation*}
  \mathscr{L} = \mathscr{L}( \Psi ^A, \Psi ^A _{,i_1}, \Psi ^A _{,i_1 i_2}, ... , \Psi ^A _{,i_1 i_2 ... i_r})
\end{equation*}
であり,
\begin{equation*}
  \Psi ^A _{,i_1 i_2 ... i_r} \equiv \dfrac{\partial ^p \Psi^A}{\partial x^{i_1} ...\partial x^{i_p}}
\end{equation*}
と定義する.このとき,EL方程式については以下のようになる.
\begin{equation*}
  \dfrac{\partial \mathscr{L}}{\partial \Psi^A} + \sum ^r _{p=1} (-1)^p \dfrac{\partial ^p}{\partial x^{i_1}\partial x^{i_2} ... \partial x^{i_p}}\left( \dfrac{\partial \mathscr{L}}{\partial \Psi ^A _{,i_1 i_2 ... i_r}} \right) =0 \tag{1.2} \label{ELeq}
\end{equation*}

基本的な発想としては通常のEL方程式の導出と大差はない.つまり,高階微分項の変分を順に取っていき微分と変分を交換し,作用積分の部分積分的操作を行って各項の一つ次元の落ちた表面項での変分がゼロと仮定して,それ以外の部分は\(\delta \Psi ^A\)でくくることができ,恒等的に作用変分がゼロであるための必要十分条件から導くことができる.

\eqref{ELeq}においては任意の高階微分まで考えているが,通常3階以上の高階微分項が現れてしまうと問題が発生するのでせっかく取った高階微分項は今回は考えずこれ以後は\(r=2\)で考える.また,一般の場の変数を考えているがここではRiemann計量\(g_{ij}\)を取る.

\subsection{この論文の構成}
なので,この論文においてはLagrangianは
\begin{equation}
  \mathscr{L} = \mathscr{L}(g_{ij}, g_{ij,k}, g_{ij,kh})
  \tag{1.3} \label{lag1}
\end{equation}
を取る.この論文の最終目標は,このようなLagrangianのもとで,次元を4次元に限定したときのEL方程式を具体的に求めたとき,Einstein方程式が唯一の方程式となることを示すことである.しかしこのようなLagrangianを用いたとき\eqref{ELeq}を見てもわかるように,4次の微分項を含んでしまう.Einstein方程式にはこれらの項は含まれないはずである.

そこで,Section2では\eqref{ELeq}の高階微分項\(g_{ij,rst}, g_{ij,rstu} \)が消えるための必要十分条件をそれぞれ求める.Section3では具体的に次元を固定し得られた必要十分条件からLagrangianの形具体的に書き下す.\(n=2,3\)の場合には,
\begin{equation*}
  \mathscr{L} = a\sqrt{g}R + b \sqrt{g}
  \tag{3.1} \label{n=2,3}
\end{equation*}
となる.\(n=4\)では,結論として得られたLagrangianから,EL方程式がEinstein方程式に一致することが導ける.
\section{Degenerate Lagrange Densities in \(n\)-Dimensionsの解説}
\subsection{準備}

\subsection{対称性について}

\subsection{EL方程式の書き換え}

\subsection{Lemma 1.}

\subsection{Lemma 2.}

\subsection{Theorem 1.}

\subsection{注意点とLemma 3.}
\section{Dimensionality Restrictionsの解説}

\subsection{Lemma 4.}

\subsection{Proof of Theorem 2.}

\subsection{Theorem 3.}

\subsection{Theorem 4.}

\subsection{Theorem 5.}

\part{The Uniqueness of the Einstein Field Equations in a Four-Dimensional}

\begin{thebibliography}{9}
\bibitem{1} 
\bibitem{2} 
\bibitem{3} 
\end{thebibliography}
\end{document}
