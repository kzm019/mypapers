\documentclass[12pt, a4paper]{jsarticle}
\usepackage[dvipdfmx]{graphicx}
\usepackage{url}
\usepackage{listings}
\usepackage{amsmath,amssymb}
\usepackage{mathtools,amssymb}
\usepackage{bm}
\usepackage{ascmac}
\usepackage{graphicx}
\usepackage[dvipdfmx]{color}
\usepackage{tikz}
\usepackage[version=4]{mhchem}
\usepackage{xcolor}
\usepackage{here}
\usepackage{wrapfig}
\usepackage{mathrsfs}
\usepackage{setspace}
\usepackage{tcolorbox}
\tcbuselibrary{theorems}
\usetikzlibrary{intersections,calc,arrows.meta}
\newcommand{\norm}[1]{\left\lVert#1\right\rVert}
\newcommand{\lam}[1]{\varLambda^{#1}}
\newcommand{\x}[1]{\chi^{#1}}

\title{ゼミ発表用ノートver.1}
\author{中越一磨}
\begin{document}
\maketitle

\section*{はじめに}
このノートは,ゼミ発表の補助として用いるために書いている.煩雑な計算をゼミ内ですべて触れることは難しいので適宜このノートを参照していただきたい.なお,各回でノートを作っていく形式ではなく更新していく形を取るので,下記途中や中途半端に書き残しているところなどお見苦しいところはあるがお許しください.

ここで,発表内容や方針などについて触れておきたい.私の卒業研究のテーマは第1回でも触れた通り,「4次元空間内での重力場方程式のEinstein方程式の唯一性」や「Gauss-Bonnet termのような高次曲率項についての考察」などとなる.そこで,これらのことを学ぶ上で以下の論文を読んでいこうと予定している.\footnote{main論文などについては今後の進捗状況に応じて変更・追加する可能性がある.}
\begin{enumerate}
  \item (main) The Einstein Tensor and its Generalizations
  \item The Uniqueness of the Einstein Field Equations in a Four-Dimensional Space
\end{enumerate}

順序としては,第2回では2.の論文を扱い,第3回で1.の論文について扱い,今後の状況に応じて第4回で1.の論文か4次元に話を絞りGauss-Bonnet項についての話題のみとするかを予定している.
\part{The Uniqueness of the Einstein Field Equations in a Four-Dimensional}
\section{Introduction}
この論文は,Einstein方程式が4次元では唯一の方程式になる事実を示したものである.
\subsection{場の解析力学}
前期にも扱いましたがLagrangianの変数を場の変数にしたときのEuler-Lagrange方程式\footnote{以後EL方程式と省略}について少し解説する.今ここでは一般に場の変数を\(\Psi^A = \Psi^A(x^i)\)として,Lagrangianは
\begin{equation*}
  \mathscr{L} = \mathscr{L}( \Psi ^A, \Psi ^A _{,i_1}, \Psi ^A _{,i_1 i_2}, ... , \Psi ^A _{,i_1 i_2 ... i_r})
\end{equation*}
であり,
\begin{equation*}
  \Psi ^A _{,i_1 i_2 ... i_r} \equiv \dfrac{\partial ^p \Psi^A}{\partial x^{i_1} ...\partial x^{i_p}}
\end{equation*}
と定義する.このとき,EL方程式については以下のようになる.
\begin{equation*}
  \dfrac{\partial \mathscr{L}}{\partial \Psi^A} + \sum ^r _{p=1} (-1)^p \dfrac{\partial ^p}{\partial x^{i_1}\partial x^{i_2} ... \partial x^{i_p}}\left( \dfrac{\partial \mathscr{L}}{\partial \Psi ^A _{,i_1 i_2 ... i_r}} \right) =0 \tag{1.2} \label{ELeq}
\end{equation*}

基本的な発想としては通常のEL方程式の導出と大差はない.つまり,高階微分項の変分を順に取っていき微分と変分を交換し,作用積分の部分積分的操作を行って各項の一つ次元の落ちた表面項での変分がゼロと仮定して,それ以外の部分は\(\delta \Psi ^A\)でくくることができ,恒等的に作用変分がゼロであるための必要十分条件から導くことができる.

\eqref{ELeq}においては任意の高階微分まで考えているが,通常3階以上の高階微分項が現れてしまうと問題が発生するのでせっかく取った高階微分項は今回は考えずこれ以後は\(r=2\)で考える.また,一般の場の変数を考えているがここではRiemann計量\(g_{ij}\)を取る.

\subsection{この論文の構成}
なので,この論文においてはLagrangianは
\begin{equation}
  \mathscr{L} = \mathscr{L}(g_{ij}, g_{ij,k}, g_{ij,kh})
  \tag{1.3} \label{lag1}
\end{equation}
を取る.この論文の最終目標は,このようなLagrangianのもとで,次元を4次元に限定したときのEL方程式を具体的に求めたとき,Einstein方程式が唯一の方程式となることを示すことである.しかしこのようなLagrangianを用いたとき\eqref{ELeq}を見てもわかるように,4次の微分項を含んでしまう.Einstein方程式にはこれらの項は含まれないはずである.

そこで,Section2では\eqref{ELeq}の高階微分項\(g_{ij,rst}, g_{ij,rstu} \)が消えるための必要十分条件をそれぞれ求める.Section3では具体的に次元を固定し得られた必要十分条件からLagrangianの形具体的に書き下す.\(n=2,3\)の場合には,
\begin{equation*}
  \mathscr{L} = a\sqrt{g}R + b \sqrt{g}
  \tag{3.1} \label{n=2,3}
\end{equation*}
となる.\(n=4\)では,結論として得られたLagrangianから,EL方程式がEinstein方程式に一致することが導ける.
\section{Degenerate Lagrange Densities in \(n\)-Dimensionsの解説}
\subsection{準備}
これ以後では,以下のことを仮定して話を進めていく.
座標変換 \(\bar{x}^i=\bar{x}^i(x^j)\)は,\(C^3\)級函数まで仮定し,そこから得られる計量は
\begin{equation*}
  \det\norm{\dfrac{\partial x^i}{\partial \bar{x}^j}}>0
\end{equation*}
のように正定値として定義する.また,議論を円滑にすすめるために以下の記号を定義する.

\begin{equation*}
  g_{ij,k} \equiv \dfrac{\partial g_{ij} }{\partial x^k } \, , \rm{etc} ...
\end{equation*}

\begin{equation*}
  \begin{split}
    \varLambda^{ij,kh} &\equiv \dfrac{\partial \mathscr{L}  }{\partial g_{ij,kh} } \\
    \varLambda^{ij,k} &\equiv \dfrac{\partial \mathscr{L}  }{\partial g_{ij,k} } \\
    \varLambda^{ij} &\equiv \dfrac{\partial \mathscr{L}  }{\partial g_{ij} }
  \end{split}
  \tag{2.2} \label{lam1}
\end{equation*}

最後に,EL方程式の左辺を
\begin{equation*}
  E^{hk} \equiv \dfrac{\partial }{\partial x^i }\left[ \varLambda^{hk,i} - \dfrac{\partial }{\partial x^j }\varLambda^{hk,ij} \right] - \varLambda^{hk}
  \tag{2.4} \label{elleft}
\end{equation*}
として,EL方程式は
\begin{equation*}
  E^{hk} =0
  \tag{2.3} \label{el}
\end{equation*}
と表せる.このとき,当然\(E^{hk}\)はそれぞれ愚直に連鎖律に基づき展開すると引数は
\begin{equation*}
  E^{hk} = E^{hk}(g_{ij}, g_{ij,r}, g_{ij,rs}, g_{ij,rst}, g_{ij,rstu})
  \tag{2.5} \label{ehk4}
\end{equation*}
となり,4次の微分項まで含んでしまう.\footnote{具体的に展開したものに関してはSection2.3以降で触れる.}このSectionの主目的はこの3次・4次の微分項の消えるための必要十分条件を探すことである.つまり,最終的に\(E^{hk}\)は,
\begin{equation*}
  E^{hk} = E^{hk}(g_{ij}, g_{ij,r}, g_{ij,rs})
  \tag{2.6} \label{ehk2}
\end{equation*}
になることを目的としている.このようになるような条件を見つけられれば, Lagrangianと\(E^{hk}\)は同じ引数を持つ函数となる.この論文では,この\(E^{hk}\)を\textit{L-degenerate}と呼ぶことにする.

\subsection{対称性について}
この項では,定義した記号の対称性について触れていく.Lagrangian密度であるためには,スカラー函数である必要がある.そのためには,以下のような恒等式を満たしている必要がある.
\begin{equation*}
  \varLambda^{ij,kh}+\varLambda^{ih,jk}+\varLambda^{ik,hj}=0
  \tag{2.7} \label{id1}
\end{equation*}
\begin{equation*}
  -\varLambda^{hk,i} = \varGamma^i_{jm} \lam{hk,jm}+2\varGamma^k_{jm} \lam{hj,im}+2\varGamma^h_{jm} \lam{kj,im}
  \tag{2.8} \label{id2}
\end{equation*}
ここで,\(\varGamma^i_{jm}\)については通常のChristoffel記号
\begin{equation*}
  \varGamma^i_{jm} = \dfrac{1}{2}g^{ih}(g_{hj,m}+g_{mh,j}-g_{jm,h})
\end{equation*}
を用いる.\eqref{id1}から次のような対称性が導ける.
\begin{equation*}
  \lam{ij,kh} = \lam{kh,ij}
  \tag{2.9} \label{sym1}
\end{equation*}
この対称性は,Bianchi恒等式からEinsteinテンソルを導いたときのように導くことができる.

\textit{Proof.} \eqref{id1}の添字をcyclicにまわしてできる4つの式を符号を交互に入れ替え辺々足し合わせていく.
\begin{align*}
  \lam{ij,kh}+\lam{ih,jk}+\lam{ik,hj}  & = 0 \\
  -\lam{hi,jk}-\lam{hk,ij}-\lam{hj,ki} & = 0 \\
  \lam{kh,ij}+\lam{kj,hi}+\lam{jh,ik}  & = 0 \\
  -\lam{jk,hi}-\lam{ji,kh}-\lam{jh,ik} & = 0
\end{align*}
このとき,偏微分の交換可能性と計量テンソルの添字の対称性を用いて,
\begin{equation*}
  2\lam{ij,kh}-2\lam{kh,ij}
\end{equation*}
より,\eqref{sym1}を得られる.\(\blacksquare\)

\subsection{EL方程式の書き換え}
この項では,具体的に4次の微分項などが消えていくための必要十分条件を求めるために\eqref{elleft}を具体的に展開していく.そのための準備として,新たに以下の記法を定義していく.
\begin{equation*}
  \lam{ij,kh;rs,tu} \equiv \dfrac{\partial \lam{ij,kh}}{\partial g_{rs,tu} }
  \tag{2.10} \label{lam2}
\end{equation*}
\begin{equation*}
  \x{ij,kh;rs,tu} \equiv \lam{ij,kh;rs,tu} + \lam{ij,ku;rs,ht} + \lam{ij,kt;rs,uh}
  \tag{2.11} \label{chi}
\end{equation*}
また,\eqref{lam2}に関して偏微分交換可能性と\eqref{sym1}により順序の入れ替えに対して以下のような対称性を持つ.
\begin{equation*}
  \lam{ij,kh;rs,tu} = \lam{rs,tu;ij,kh} = \lam{ij,kh;tu,rs}
  \tag{2.12} \label{sym2}
\end{equation*}

以上で,準備は整ったので具体的に展開をしていく.Lagrangianは仮定より2次の微分項までしか持たないので,\(\mathscr{L} = \mathscr{L}(g_{mn}, g_{mn,p}, g_{mn,pq})\)となる.このことから,\(E^{hk}\)のそれぞれの項を連鎖律に従い展開していく.

はじめに,\(\partial \lam{hk,i} / \partial x^i\)は
\begin{align*}
  \frac{\partial}{\partial x^{i}}\left(\frac{\partial \mathcal{L}}{\partial g_{hk,i}}\right) & = \frac{\partial g_{m n}}{\partial x^{i}} \frac{\partial^{2} \mathcal{L}}{\partial g_{m n} \partial g_{h k, i}}+\frac{\partial q_{m n, p}}{\partial x^{i}} \frac{\partial^{2} \mathcal{L}}{\partial g_{m n, p} \partial g_{h k, i}} +\frac{\partial g_{m n, p q}}{\partial x^{i}} \frac{\partial^{2} \mathcal{L}}{\partial g_{m n, p q} \partial g_{h k, i}} \\
                                                                                             & = \lam{hk,i;mn} g_{mn,i} +\lam{hk,i;mn,p} g_{mn,pi}+\lam{hk,i;mn,pq} g_{mn,pqi}
\end{align*}
のようになる.同様にして,\(\partial^2 \lam{hk,ij} / \partial x^i \partial x^j\)の項は,
\begin{align*}
   & \frac{\partial^2}{\partial x^{i}\partial x^{j}}\left(\frac{\partial \mathcal{L}}{\partial g_{hk,i}}\right)                                                                                                     \\
   & = \lam{hk,ij;mn} g_{mn,ij}                                                                                 & + & \lam{hk,ij;mn,p} g_{mn,pij}                & + & \tcboxmath[
    colback=cyan!8,   % 背景の色
    colframe=cyan,     % 枠の色
    sharp corners,     % 角を直角に
  ]{
    \lam{hk,ij;mn,pq} g_{mn,pqij}
  }                                                                                                                                                                                                                 \\
   & + \lam{hk,ij;mn;rs} g_{mn,i}g_{rs,j}                                                                       & + & \lam{hk,ij;mn;rs,t} g_{mn,i}g_{rs,tj}      & + & \lam{hk,ij;mn;rs,tu} g_{mn,i}g_{rs,tuj}      \\
   & + \lam{hk,ij;mn,p;rs} g_{mn,pi}g_{rs,j}                                                                    & + & \lam{hk,ij;mn,p;rs,t} g_{mn,pi}g_{rs,tj}   & + & \lam{hk,ij;mn,p;rs,tu} g_{mn,pi}g_{rs,tuj}   \\
   & + \lam{hk,ij;mn,pq;rs} g_{mn,pqi}g_{rs,j}                                                                  & + & \lam{hk,ij;mn,pq;rs,t} g_{mn,pqi}g_{rs,tj} & + & \lam{hk,ij;mn,pq;rs,tu} g_{mn,pqi}g_{rs,tuj}
\end{align*}
のようになる.これらの計算から4次の微分項は青いボックスで囲まれたところのみとなり,3次以下の微分の項を今は\(O^{hk} = O^{hk}(g_{ij},\: g_{ij,r}, \: g_{ij,rs},\: g_{ij,rst})\)として
\begin{equation*}
  E^{hk} =- \lam{hk,ij;rs,tu} g_{rs,tuij}+O^{hk}
  \tag{2.13} \label{EL2}
\end{equation*}
と書き換えることができる.
\subsection{Lemma 1.}
先の書き換えの結果から,\(E^{hk}\)の \(g_{rs,tuij}\)のみを取り出すことができた.また,\(E^{hk}\)の中に4次の項を含まないということは,
\begin{equation*}
  \dfrac{\partial E^{hk}}{\partial g_{rs,tuij} } = 0
\end{equation*}
が成り立つと言い換えられる.このことから,安直にその係数がゼロになる,つまり
\begin{equation*}
  \lam{hk,ij;rs,tu}=0
\end{equation*}
としてしまうとここまでの計算が台無しになってしまう.

注意すべきは,計量の4次の微分項に関しては\((tuij)\)の添字はどれを交換しても偏微分は交換可能であると仮定しているため対称であり,さらに今~\eqref{EL2}の通りそれぞれの添字は縮約を取っているということだ.このようなことを留意すれば,どのような順序の添字で微分してもしっかりと消去可能であるための条件は以下のようになる.
\begin{align*}
  \dfrac{\partial E^{hk}}{\partial g_{rs,tuij} }
  \simeq &
  \dfrac{\partial }{\partial g_{rs,tuij}  }
  (( \lam{hk,ij;rs,tu} + \lam{hk,iu;rs,jt} + \lam{hk,it;rs,uj}                      \\
         & +\lam{rs,tu;hk,ij} + \lam{rs,tj;hk,ui} + \lam{rs,ti;hk,ju}) g_{rs,tuij}) \\
  =      & \x{hk,ij;rs,tu}+\x{rs,ij;hk,tu}=0
\end{align*}

\textbf{Lemma 1.}\:\(E^{hk}\)が
\begin{equation*}
  \dfrac{\partial E^{hk}}{\partial g_{rs,tuij} } = 0
\end{equation*}
のようになるための必要十分条件は
\begin{equation*}
  \x{hk,ij;rs,tu}= -\x{rs,ij;hk,tu}
  \tag{2.14} \label{lemma1}
\end{equation*}
である.
\subsection{Lemma 2.}
Section2.3での計算をもとに,\(O^{hk}\)の3次の微分項を抽出していく.それ以下の項を\(P^{hk}=P^{hk}(g_{ij}, g_{ij,r},g_{ij,rs})\)とする.

\begin{equation*}

\end{equation*}
\subsection{Theorem 1.}

\subsection{注意点とLemma 3.}
\section{Dimensionality Restrictionsの解説}

\subsection{Lemma 4.}

\subsection{Proof of Theorem 2.}

\subsection{Theorem 3.}

\subsection{Theorem 4.}

\subsection{Theorem 5.}

\part{The Uniqueness of the Einstein Field Equations in a Four-Dimensional}

\begin{thebibliography}{99}
  \bibitem{aboutLaTeX} D.Lovelock, The Uniqueness of the Einstein Field Equations in a Four-Dimensional Space, Archive for Rational Mechanics and Analysis, 1, 54--70 (1969)
\end{thebibliography}

\end{document}
