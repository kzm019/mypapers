\documentclass[12pt, a4paper]{jsarticle}
\usepackage[dvipdfmx]{graphicx}
\usepackage{url}
\usepackage{listings}
\usepackage{amsmath,amssymb}
\usepackage{mathtools,amssymb}
\usepackage{bm}
\usepackage{ascmac}
\usepackage{graphicx}
\usepackage[dvipdfmx]{color}
\usepackage{tikz}
\usepackage[version=4]{mhchem}
\usepackage{xcolor}
\usepackage{here}
\usepackage{wrapfig}
\usetikzlibrary{intersections,calc,arrows.meta}

\title{ゼミ発表用ノートver.1}
\author{中越一磨}
\begin{document}
\maketitle

\section{はじめに}
このノートは,ゼミ発表の補助として用いるために書いております.煩雑な計算をゼミ内ですべて触れることは難しいので適宜このノートを参照していただければ幸いです.なお,各回でノートを作っていく形式ではなく更新していく形を取るので,下記途中や中途半端に書き残しているところなどお見苦しいところはありますでしょうがお許しください.

ここで,発表内容や方針などについて触れておきたいと思います.私の卒業研究のテーマは第1回でも触れた通り,「4次元空間内での重力場方程式のEinstein方程式の唯一性」や「Gauss-Bonnet termのような高次曲率項についての考察」などです.そこで,これらのことを学ぶ上で以下の論文を読んでいこうと予定しています.\footnote{main論文などについては今後の進捗状況に応じて変更・追加する可能性があります.}
\begin{enumerate}
  \item (main) The Einstein Tensor and its Generalizations
  \item The Uniqueness of the Einstein Field Equations in a Four-Dimensional Space
\end{enumerate}

順序としては,第2回では2.の論文を扱い,第3回で1.の論文について扱い,今後の状況に応じて第4回で1.の論文か4次元に話を絞りGauss-Bonnet項についての話題のみとするかを予定しています.
\part{The Uniqueness of the Einstein Field Equations in a Four-Dimensional}
\section{Introduction}

\section{Degenerate Lagrange Densities in \(n\)-Dimensionsの解説}

\section{Dimensionality Restrictionsの解説}

\part{The Uniqueness of the Einstein Field Equations in a Four-Dimensional}

\begin{thebibliography}{9}
\bibitem{1} 
\bibitem{2} 
\bibitem{3} 
\end{thebibliography}
\end{document}
